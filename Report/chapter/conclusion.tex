\chapter{Conclusion}
\label{chp:conc}

Splitting the training up into more batches would probably be preferable for a design build from the button up and running it with a GPU, but this was not the scope of this project and so the CPU solution was preferred. The same scenario happened when the images had to be rotated and tested in the model, just in this case, all the models were not able to run on the GPU. All rotational results were made with a slower solution by using the CPU, but it was possible to do so with the extra memory.


An extra two percentages can sometimes be found by using model ensemble, but it was not in the scope of this project to optimize the models, but simply to compare the results.

Generally DenseNet is the best network for handling rotation invariance.