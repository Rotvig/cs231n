\chapter{Theory}
\label{chp:theory}

This chapter describes the theories behind RegularNet, ResNet and DenseNet. Furthermore Feature Invariance is elaborated, as it is used to explore the classification capabilities of RegularNet, ResNet and DenseNet.

\section{Feature Invariance}
%https://drive.google.com/drive/folders/0Bw37nIXex8aXZC02T2FDdk1IaTA
A neural network can result in a test accuracy close to or above the accuracy of human decisions. A trained neural network model can therefore be used as an approved classification tool. When the classification tool is introduced to the industry the misclassification rate, can be too high. This could happen because of feature invariance in the images used for classification in the model. Feature invariance testing is a mayor scope of this project, and it can be done in several ways. Feature invariance testing can be done by transforming, scaling, occlusion and rotation of images. Figure \ref{fig:FeatureInvarianceSummary} shows some of the different feature variances an image can have.

\myFigure{FeatureInvariance_summary.png}{A summary of different variances in image features}{fig:FeatureInvarianceSummary}{0.5}
\FloatBarrier

When transforming or scaling test images to verify the accuracy, interpolation can fill out the missing information and so the model might score a bit less in probability for certain labels. The predicted true label is, most likely, still higher than other labels and so this type of image variance often test well for invariance testing. This is also verified in the article \textit{Visualizing and Understanding Convolutional Network} by \citet{Visualizing}, as seen in figure \ref{fig:FeatureInvariancePhotoseries}. In \citet{Visualizing}, they test a model against some test images, which have been transformed, scaled and rotated. The rotation of an image rearranges the pixels in another way than simple interpolation can solve. Instead of a sky always being in top of the image, the sky is in the side of the image and so rotation invariance test will give different arrangement of the pixels than the feature maps are used to. The robustness of the model is tested in this project based on the rotation invariance of the model.

\myFigure{FeatureInvariance_Photoseries.png}{Photoseries feature variances, rotation, in an image}{fig:FeatureInvariancePhotoseries}{0.5}
\FloatBarrier

\section{Deep Neural Net - RegularNet}
\todo{Starten på dette afsnit virker lidt underligt?}composed exclusively of regular and strided convolutional layers. While this architecture works well for relatively shallow networks, it becomes increasingly more difficult to train as the network depth increases.

Regular networks are a combination of the two standard forms of neural networks, the fully connected neural network and the convolution neural network.

\subsection{Fully Connected Neural Networks (FCNN)}
A neural network consist of an input layer, some hidden layers and an output layer. The input layer is an image, and it is targeted for feature extraction and classification in the hidden layers and output layer. Every pixel in the image is connected to every neuron in the first hidden layer of the neural network as seen in figure \re{fig:RegularFCNN}.    
Each hidden layer is an array of neurons, with each neuron normally consisting of a weight, some activation function and a regulation function. Each pixel value is weighted, activated in an activation function and regularized. Most often ReLU or Leaky ReLU are used as activation function to zero out negative values. The activation function provides a non-linear relationship within the data points to provide better feature extractions. Each neuron in the second hidden layer is supplied with the output of all the neurons in the previous layer and the procedure is repeated until the last hidden layer. The output layer consist of a loss function which is often either a Softmax loss function or a support vector machine loss function. The amount of loss functions in the output layer is equal to the amount of classification categories the image can be classified as.

An image is forwarded through the neural network, and the loss functions provides the misclassification percentage of the image. This error, or loss in accuracy, is send back through the neural network and a gradient for each neuron is found. This process is repeated for, normally, an integer amount of images, batch size, and for each backward propagation the gradient is saved. After a batch of images all the gradients saved for an individual neuron are average and from this value the neural network response to the back propagation. The weights are updated depending on the averaged gradients and this procedure can be done in different ways. The most modern update method is called ADAM, and it is trying to reduce the loss result by changing the weights regarding to the gradients. ADAM will decide the amount of change the weights must have in relation to the gradients, while the regularization step decides the relationship of the change between the weights. ADAM defines the maximum change to a single weight, while the regularization defines the distribution of change over all weights related to the maximum defined change.

\myFigure{RegularNet_FCNN.png}{Fully Connected Neural Network Architecture \citep{RegularNET_FCNN}}{fig:RegularFCNN}{0.5}
\FloatBarrier

\subsection{Convolutional Neural Networks (CNN)}

The main difference between the FCNN and CNN is the FCNN is providing every input pixel to every neuron between each layer, while the CNN is only connecting the neurons to regions of interest of the image, also known as receptive fields. Each receptive field is dotted together with a feature map which provides a single pixel output.

The input layer is an image, while the next layer consist of weighted feature maps, an activation function and a regulation function. Each feature map consist of three spatial dimensions. One spatial dimension of the feature map is convolved with one spatial dimension of the image. The result of dot products for the image pixel at one receptive field with the feature map provides a value for the receptive field. Convolving the input image of \todo{Hvad er N og M i denne sammenhæng, forklar gerne i teksten hvad det er?} $NxMx3$ with X feature maps of 3x3x3 provides X new images with a dimension determine by the striding distance for the convolution as seen in figure \ref{fig:RegularStrinding}.

\myFigure{RegularNET_Strinding.png}{Convolutional Neural Network feature map striding example \citep{RegularNET_Strinding}}{fig:RegularStrinding}{0.5}
\FloatBarrier

Pooling layers are put in between layers at convenient places, to minimize variable count and training time for the neural network. This is typical done by increasing filter strides or by max pooling. Max pooling is choosing the highest pixel value in a spatial dimension of the image and compressing the images to only the chosen max pooling values, as seen in figure \ref{fig:RegularCNN}. It will not always be possible to have the desired stride unit distance in the image, so an extra layer of zeros are padded on to the image to make sure the image is filtered with the correct feature map size and stride unit distance. One zero padding layer and an 3x3 feature map will return an image with the same size as the input image.

\myFigure{RegularNet_CVNN.png}{Convolutional Neural Network Architecture \citep{RegularNET_CVNN}}{fig:RegularCVNN}{0.5}
\FloatBarrier
\newpage
\section{Deep Residual Networks - ResNet}
%Peter

Deep convolutional neural networks is today used to produce the best results on the ImageNet dataset, and this reveals that the depth of the networks is of high importance for the performance of the networks\citep{RESNET}. Some of the leading results on the ImageNet dataset have models with a depth of 16 to 30 layers. The real question is if learning better networks is as easy as having more layers in the network.

\myFigure{plain_network}{Results from two networks run on the CIFAR-10 dataset with 20 and 56 layers.\citep{RESNET}}{fig:plain}{1}

On figure \ref{fig:plain} it is shown that a convolutional neural network with 20 layers achieves a better performance than a network with 56 layers. A reason why the 56-layer network is bad could be the vanishing / exploding gradients problem, which hamper convergence from the beginning of the training. This problem can be solved by normalized initialization and intermediate normalization layers, which will make the network start converging. After solving this problem the deeper networks will be able to converge, but here a degradation problem might occur, meaning that with the network depth increasing the accuracy gets saturated. the degradation problem is also shown on figure \ref{fig:plain}. To solve this problem a deep residual network can be used.

\myFigure{res_block}{A residual building block. \citep{RESNET}}{fig:resblock}{0.5}

Figure \ref{fig:resblock} shows a residual learning building block. In a residual neural network the underlying layers are fit a residual mapping. The underlying mapping is referred to as \emph{H(x)}, where the nonlinear layers fit a mapping of \emph{F(x) = H(x) - x}. The original mapping is represented as \emph{F(x) + x}, where this mapping can be realized by feed forward neural networks with shortcut connections. Shortcut connections are connections which takes the input \emph{x} and skips it forward to the output of the stacked layers, as seen in figure ref{fig:resblock}, this is also called identity mapping. The shortcut connections does not add extra complexity to the network. Each residual building block can be defined as:

\begin{equation} \label{eq:res}
y = F(x, {Wi}) + x
\end{equation} 

In equation \ref{eq:res} x and y are the input and output vectors of the layers in the building block. The $F(x,{Wi})$ function represents the residual mapping to be learned throughout the layers. By using figure \ref{fig:resblock} as a building block, the following function will be used for the two layers: 

\begin{equation} \label{eq:func_res}
F = W2\sigma(W1x)
\end{equation} 

In equation \ref{eq:func_res} $\sigma$ is the ReLU activation function, while W2 and W1 is the weights for each layer. The shortcut connections added in equation \ref{eq:res} does not introduce any extra parameters or computation complexity. This is good for the comparison between plain and residual networks, as two networks can be compared easily as they have the same amount of parameters, depth, width and computational cost.
\newline

On figure \ref{fig:plainvsres} the first layers of a plain and a residual network is shown. The plain network consist of convolutional layers with a 3x3 filter. When the output feature maps are the same size, the layers have the same amount of filters. If the size of the feature map is divided by two, the filters will be multiplied with two, and this will preserve the time complexity per layer. Downscaling is performed by convolutional layers that have a stride 2.

\myFigure{plain_vs_res}{The left shows the first layers of a plain 34-layer network. The right shows the same layers from a 34-layer residual network. \citep{RESNET}}{fig:plainvsres}{0.5}
\FloatBarrier

The difference from the plain network to the residual network architecture is the shortcut connections. When the input and the output of the layers are of the same dimensions the identity shortcuts can be used directly, this is shown on figure \ref{fig:plainvsres} by the solid lines going from input to output. When the dimensions between input and output increase the dotted line is used in figure \ref{fig:plainvsres}. Two solutions exist to the shortcut connections between two different input and output dimensions. The first solution is to pad extra zeros to the input image, to achieve the same dimension as the output image. This solution does not add any extra paramters to equation \ref{eq:res}. Another solution is to use a projection shortcut which is used to match the dimensions between input and ouput done by 1x1 convolutions. Adding the projection shortcut gives the equation shown in equation \ref{eq:func_proj}, where Ws is added to the equation.

\begin{equation} \label{eq:func_proj}
y = F(x, {Wi}) + Wsx.
\end{equation} 

\section{Densely Connected Convolutional Networks - DenseNet}




\myFigure{dens.PNG}{A deep Densely Connected Convolutional Network where there are three dense blocks. The blocks between the three dense blocks are adjacent blocks also referred to as transition layers\citep{DENSE}}{fig:architecture}{1}

\myFigure{denselayers.PNG}{A 5 layer dens block with \citep{DENSE}}{fig:dense}{1}
