\chapter{Theory}
\label{chp:theory}

This chapter describes the theories behind RegularNet, ResNet and DenseNet. Furthermore Feature Invariance is elaborated as it is used to explore the classification capabilities of RegularNet, ResNet and DenseNet.

\section{Feature Invariance}
%https://drive.google.com/drive/folders/0Bw37nIXex8aXZC02T2FDdk1IaTA
A neural network can result in a test accuracy which exceeds all expectations and everybody is happy. Now, when the model is introduced to the industry the misclassification rate happened to be high. This could happen, because of a feature invariance in the images used to train the model. Feature invariance testing is a mayor scope of this project. Feature invariance testing can be done by transforming, scaling, occlusion and rotating images just to name a few, while in this project an investigation of the latter is done. Figure fig:FeatureInvarianceSummary shows fine the different features variances an image could have.

\myFigure{FeatureInvariance_summary.png}{A summary of different variances in image features \citep{FeatureInvariance_summary.png}}{fig:FeatureInvarianceSummary}{0.5}
\FloatBarrier

When transforming or scaling test images to verify the accuracy, interpolation can fill out the missing information and so the model might score a bit smaller in probability for a certain category, but it is most likely still higher than other categories and so this type of image variance often test well for invariance testing. This is also verified in RefToVisualizationArticle, as seen in figure FeatureInvariancePhotoseries, in which they test their model against test images, which have been transformed, scaled and rotated. The rotation of an images rearranges the pixels in another way than simple interpolation can solve. Instead of a sky always being in top now the sky is in the side and so rotation invariance test can be tough for the model to chew. The robustness of the model is tested in this project based on the rotation invariance of the model.

\myFigure{FeatureInvariance_Photoseries.png}{Photoseries feature variances, rotation, in an image. \citep{FeatureInvariance_Photoseries.png}}{fig:FeatureInvariancePhotoseries}{0.5}
\FloatBarrier

\section{Deep Neural Net - RegularNet}
composed exclusively of regular and strided convolutional layers. While this architecture works well for relatively shallow networks, it becomes increasingly more difficult to train as the network depth increases.

Regular networks are a combination of the two standard forms of neural networks, the fully connected neural network and the convolution neural network.

\subsection{Fully Connected Neural Networks}
A neural network consist of an input layer, some hidden layers and an output layer. The input layer is an image, and it is targeted for feature extraction and classification in the hidden layers and output layer. Every pixel in the image is connected to every neuron in the first hidden layer of the neural network.    
Each hidden layer is an array of neurons, with each neuron normally consisting of a weight, some activation function and a regulation function. Each pixel value is so weighted, activated in an activation function and regularized. Most often ReLU or Leaky ReLU are used as activation function to zero out negative values. The activation function provides a non-linear relationship within the data points to provide better feature extractions. Each neuron in the second hidden layer is supplied with the output of all the neurons in the previous layer and the procedure is repeated until the last hidden layer. The output layer consist of a loss function which is often either a Softmax loss function or a support vector machine loss function. The amount of loss functions in the output layer is equal to the amount of classification categories the image can be classified as.

An image is forwarded through the neural network, and the loss functions provided the misclassification percentage of the image. This error, or loss in accuracy, is send back through the neural network and a gradient for each neuron is found. This process is repeated for, normally, an integer amount of images, batchsize, and for each backward propagation the gradient is saved. After a batch of images all the gradients saved for an individual neuron are average and from this value the neural network response to the back propagation. The weights are updated depending on the averaged gradients and this procedure can be done in different ways. The most modern update method is called ADAM, and it is trying to reduced the loss result by changing the weights regarding the gradients. ADAM will decided the amount of change the weights must have in relation to the gradients, while the regularization step decides the relationship of the change between the weights. Say ADAM defines the maximum change to a single weight, while the regularization defines the distribution of change over all weights related to the maximum defined change.

\myFigure{Regular}{Fully Connected Neural Network Architecture \citep{RegularNET_FCNN}}{fig:RegularFCNN}{0.5}
\FloatBarrier

\subsection{Convolutional Neural Networks}


%Troels

\section{Deep Residual Networks - ResNet}
%Peter

Deep convolutional neural networks is today used to produce the best results on the ImageNet dataset, and this reveals that the depth of the networks is of high importance for the performance of the networks\citep{RESNET}. Some of the leading results on the ImageNet dataset have models with a depth of 16 to 30 layers. The real question is if learning better networks is as easy as having more layers in the network.

\myFigure{plain_network}{Results from two networks run on the CIFAR-10 dataset with 20 and 56 layers.\citep{RESNET}}{fig:plain}{1}

On figure \ref{fig:plain} it is shown that a convolutional neural network with 20 layers achieves a better performance than a network with 56 layers. A reason why the 56-layer network is bad could be the vanishing / exploding gradients problem, which hamper convergence from the beginning of the training. This problem can be solved by normalized initialization and intermediate normalization layers, which will make the network start converging. After solving this problem the deeper networks will be able to converge, but here a degradation problem might occur, meaning that with the network depth increasing the accuracy gets saturated. the degradation problem is also shown on figure \ref{fig:plain}. To solve this problem a deep residual network can be used.

\myFigure{res_block}{A residual building block. \citep{RESNET}}{fig:resblock}{1}

Figure \ref{fig:resblock} shows a residual learning building block. In a residual neural network the underlying layers are fit a residual mapping. The underlying mapping is reffered to as \emph{H(x)}, where the nonlinear layers fit a mapping of \emph{F(x) = H(x) - x}. The original mapping is represented as \emph{F(x) + x}, where this mapping can be realized by feedforward neural networks with shortcut connections. Shortcut connections are connections which takes the input \emph{x} and skips it forward to the output of the stacked layers, as seen in figure ref{fig:resblock}, this is also called identity mapping. The shortcut connections does not add extra complexity to the network.

\section{Densely Connected Convolutional Networks - DenseNet}

A traditional convolutional networks consists of L layers and have L connections - one between each layer and its subsequent layer. The DenseNet also known as Densely Connected Convolutional Networks has $L(L+1)/2$ direct connections. The DenseNet has several compelling advantages. For example it alleviates the vanishing-gradient problem, it strengthen feature propagation and substantially reduce the number of parameters.

A DenseNet does not solely consist of dense blocks. It also consists of convolution, pooling and some classification layer at the end. This can be seen in figure \ref{fig:architecture}. The convolution and pooling layer are referred to as the transition layers.

\myFigure{dens.PNG}{A deep densely connected convolutional network where there are three dense blocks. The blocks between the three dense blocks are adjacent blocks also referred to as transition layers\citep{DENSE}}{fig:architecture}{1}

To improve the flow of information in the DenseNet - direct connection is used from any layer to all subsequent layers. The results of this is that the $l^{th}$ layer receives the feature-map of all the past layers. The output of the $l^{th}$ layer is denoted as $X_l$.

\begin{equation}
X_l=H_l([x_0,x_1,...,X_{l-1}])
\end{equation}

$[x_0,x_1,...,X_{l-1}]$ refers to the concatenation of the feature-maps, which the layers produces. This network architecture is referred to as DenseNet. The direct connection is illustrated in figure \ref{fig:dense}. The multiple inputs of $H_e(.)$ are concatenated into a single tensor\footnote{Tensors are geometric objects that describe the linear relation between geometric vectors, scalars and other tensors}.

Convolutional networks's becomes deeper. With the increasing depth new problem emerges. Such as the information of the input or gradient are passed through many layers, it can vanish this is also known as the vanishing-gradient problem. A solution for this is to bypass the signal from one layer to the next. The solution in DenseNet is to connect all layers directly with each other. This connection preserves the feed-forward nature and each layers receives additional inputs from all preceding layers. This is seen in figure \ref{fig:dense}.

\myFigure{denselayers.PNG}{A 5 layer dens block \citep{DENSE}}{fig:dense}{0.7}

In figure \ref{fig:dense} a 5 layers dens block is illustrated. Between each of the layers in the dense block \emph{Batch Normalization}, \emph{ReLU} and convolutional layer are applied.
