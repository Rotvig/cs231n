\chapter{Implementation}
\label{chp:impl}

This chapter describes some of the implementation details of the project. The idea of the project is to see the classification capabilities of RegularNet, ResNet and DenseNet when the original images has been rotated from 0 to 360 degrees. The implementation of the different nets are implemented in a Jupyter Notebook.

\section{Tensorflow}
TensorFlow is used in this project to create RegularNet, ResNet and DenseNet. 
TensorFlow is an open-source software library created to ease the implementation and use of neural nets. The code base is on github\footnote{\url{https://github.com/tensorflow/tensorflow}}.
%https://www.draw.io/?lightbox=1&highlight=0000ff&edit=_blank&layers=1&nav=1#G0Bw37nIXex8aXYmE1Mll1U1oxUmc
\myFigure{tensorflow.PNG}{Illustration of how TensorFlow is used in this project}{fig:tf}{0.8}

In figure \ref{fig:tf} it is seen how TensorFlow is used in this project. The first part of using TensorFlow is to create the neural net that is needed - how this is done is elaborated in the following chapters. When the neural net is created in TensorFlow the training is ready to commence. 

The Training then yields a model which can be used for classification of images. The model is tested with Cifar-10 test batch. The Output contains probabilities of the classifications - such as 10 \% probability that the image is a truck.
\section{Feature Invariance}

\section{Deep Neural Net - RegularNet}
Composed exclusively of regular and strided convolutional layers. While this architecture works well for relatively shallow networks, it becomes increasingly more difficult to train as the network depth increases.

This section will elaborate on the implementaiton of a RegularNet used for this project. For the implementation python 2.7 and Jupyter is used. The implementation of the RegularNet is from the github repository\footnote{\url{https://github.com/awjuliani/TF-Tutorials}}. 

The RegularNet implemented in the project consists of 34 layers, with an image as input layer. The first five layers in the net are normal convolutional layers with a zero padding of one, striding unit distance of one and feature mapping of 3x3. The convolutional layers are followed by a convolutional layer with a striding unit distance of two. The sixth layer functions as a pooling layer since the output from the layer is an image of half the input size. The RegularNet consists of 25 convolutional layers, 5 pooling layers, 1 input layer, 1 flatten layer and a softmax classifier layer is appended to the end of the RegularNet. Adam is used as the activation function for the RegularNet, and normalization is used for regularization function. The creation of the RegularNet layers can be seen in listing \ref{lst:regularloop}.

\begin{lstlisting}[language=Python, label=lst:regularloop, caption=For loop that creates the layers in the RegularNet]
input_layer = tf.placeholder(shape=[None,32,32,3],dtype=tf.float32,name='input')
label_layer = tf.placeholder(shape=[None],dtype=tf.int32)
label_oh = slim.layers.one_hot_encoding(label_layer,10)

layer1 = slim.conv2d(input_layer,64,[3,3],normalizer_fn=slim.batch_norm,scope='conv_'+str(0))
for i in range(5):
for j in range(units_between_stride):
layer1 = slim.conv2d(layer1,64,[3,3],normalizer_fn=slim.batch_norm,scope='conv_'+str((j+1) + (i*units_between_stride)))
layer1 = slim.conv2d(layer1,64,[3,3],stride=[2,2],normalizer_fn=slim.batch_norm,scope='conv_s_'+str(i))

top = slim.conv2d(layer1,10,[3,3],normalizer_fn=slim.batch_norm,activation_fn=None,scope='conv_top')

output = slim.layers.softmax(slim.layers.flatten(top))
\end{lstlisting}

As seen in listing \ref{lst:regularloop} the function conv2d is called to create a convolution neuron. This function creates the convolution block with the internal layers and returns the output from the block. One neuron consists of 1 convolutional layer and 1 batch normalization block. Between the convolutional layer and the batch layer a ReLU activation function is added. The convolutional layer consists of 64 feature maps which each has 9 weights and so the convolutional layer outputs 64 new images for the relu function. The code for creating a residual block can be seen in listing \ref{lst:ConvLoss}.

\begin{lstlisting}[language=Python, label=lst:ConvLoss, caption= Implementation of learning rate type ADAM]
loss = tf.reduce_mean(-tf.reduce_sum(label_oh * tf.log(output) + 1e-10, axis=[1]))
trainer = tf.train.AdamOptimizer(learning_rate=0.001)
update = trainer.minimize(loss)
\end{lstlisting}

\section{Deep Residual Networks - ResNet}

This section will elaborate on the implementaiton of ResNet used for this project. For the implementation python 2.7 and Jupyter is used. The implementation of the ResNet is from the github repository\footnote{\url{https://github.com/awjuliani/TF-Tutorials}}. 

The ResNet implemented in the project consists of 34 layers, where a flatten layer and a softmax classifier layer is appended to the end of the ResNet. The first layer in the net is a normal convolutional layer. 25 of the layers are residual blocks, which also contains internal layers. After 5 residual blocks a pooling layer is added with a 3x3 filter and a stride with 2x2, meaning that a total of 5 pooling layers will be added to the net. At last a fully connected layer is added to the top with the flatten and softmax classifier. Adam is used as the activation function for the ResNet. The creation of the ResNet layers can be seen in listing \ref{lst:resloop}.

\begin{lstlisting}[language=Python, label=lst:resloop, caption=For loop that creates the layers in the ResNet]
layer1 = slim.conv2d(input_layer,64,[3,3],normalizer_fn=slim.batch_norm,scope='conv_'+str(0))
for i in range(5):
    for j in range(units_between_stride):
        layer1 = resUnit(layer1,j + (i*units_between_stride))
    layer1 = slim.conv2d(layer1,64,[3,3],stride=[2,2],normalizer_fn=slim.batch_norm,scope='conv_s_'+str(i))
    
top = slim.conv2d(layer1,10,[3,3],normalizer_fn=slim.batch_norm,activation_fn=None,scope='conv_top')

output = slim.layers.softmax(slim.layers.flatten(top))
\end{lstlisting}

As seen in listing \ref{lst:resloop} the function resUnit is called to create a residual block. This function creates the residual block with the internal layers and returns the output from the block. The residual block consists of 2 batch normalization blocks and 2 convolutional layers. Between the convolutional layer and the batch layer a ReLU activation function is added. At last the output from the layers is added with the input, and this is how the shortcut connection is implemented. The code for creating a residual block can be seen in listing \ref{lst:resblock}.

\begin{lstlisting}[language=Python, label=lst:resblock, caption=Function resUnit that creates the residual block]
def resUnit(input_layer,i):
    with tf.variable_scope("res_unit"+str(i)):
        part1 = slim.batch_norm(input_layer,activation_fn=None)
        part2 = tf.nn.relu(part1)
        part3 = slim.conv2d(part2,64,[3,3],activation_fn=None)
        part4 = slim.batch_norm(part3,activation_fn=None)
        part5 = tf.nn.relu(part4)
        part6 = slim.conv2d(part5,64,[3,3],activation_fn=None)
        output = input_layer + part6
        return output
\end{lstlisting}

When training neural networks the distribution of each layers input change, because of the parameters of the previous layers change\citep{BATCH}. This makes it hard to train models with nonlinearities as the training is slowed down because of a lower learning rate and careful paramter initialization. The solution to this problem is the batch normalization layers, which makes normalization a part of the nets architecture, by performing a normalization for each training batch. Batch normalization makes it possible to use a higher learning rate, and gives the freedom to care less about initialization. Batch normalization aslo acts as a regularier, and can in some cases eliminate the need for dropout.



\section{Densely Connected Convolutional Networks - DenseNet}

This section will elaborate on the implemented DenseNet. The implementation is done in python 2.7 and Jupyter. The implementation of the DenseNet is from the github repository\footnote{\url{https://github.com/awjuliani/TF-Tutorials}}. Article \citep{DENSE} has been used as the basis for the implementation of the DenseNet.

The DenseNet in this project consist of 12 layers where 5 layers are dense blocks, and 7 layers are convolutional layers. At the end of the DenseNet a flatten layer and a softmax layer are applied. Adam is used as activation function.

The creation of the 7 convolutional layers and 5 dense block's can be seen in listing \ref{lst:denseconvo}.

\begin{lstlisting}[language=Python, label=lst:denseconvo, caption=for loop which creates the dense block's mixed with the convolutional layers]
layer1 = slim.conv2d(input_layer,64,[3,3],
	normalizer_fn=slim.batch_norm,scope='conv_'+str(0))
for i in range(5):
	layer1 = denseBlock(layer1,i,units_between_stride)
	layer1 = slim.conv2d(layer1,64,[3,3],stride=[2,2],
		normalizer_fn=slim.batch_norm,scope='conv_s_'+str(i))

top = slim.conv2d(layer1,10,[3,3],
	normalizer_fn=slim.batch_norm,activation_fn=None,scope='conv_top')
\end{lstlisting}

The convolutional layers has a filter size of 3 by 3, and a stride with 2 by 2. 

The dense block's consist of 6 convolutional layers which are connected in a direct connection to all subsequent layers and have a filter size of 3 by 3. The code for creating the dense block's can be seen in listing \ref{lst:densenetblock}. The last convolutional layer is the top layer which is a fully connected layer.

\begin{lstlisting}[language=Python, label=lst:densenetblock, caption=DenseNet Block function]
def denseBlock(input_layer,i,j):
	with tf.variable_scope("dense_unit"+str(i)):
	nodes = []
	a = slim.conv2d(input_layer,64,[3,3],normalizer_fn=slim.batch_norm)
	nodes.append(a)
	for z in range(j):
		b = slim.conv2d(tf.concat(nodes,3),64,[3,3],
				normalizer_fn=slim.batch_norm)
		nodes.append(b)
	return b
\end{lstlisting}

Listing \ref{lst:softmax} shows the last layer which is the softmax layer. This layer is appended to \emph{top} which contains all the previous layers.

\begin{lstlisting}[language=Python, label=lst:softmax, caption=Softmax layer appended to the end of the DenseNet]
output = slim.layers.softmax(slim.layers.flatten(top))
\end{lstlisting}

When all the layers are prepared the DenseNet is ready to be used with TensorFlow, and the training can commence.
