\chapter{Implementation}
\label{chp:impl}

This chapter describes some of the implementation details of the project. The idea of the project is to see the classification capabilities of RegularNet, ResNet and DenseNet when the original images has been rotated from 0 to 360 degrees. The implementation of the different nets are implemented in a Jupyter Notebook.

\section{Tensorflow}
Tensorflow is used in this project to create RegularNet, ResNet and DenseNet. 
Tensorflow is an open-source software library created to ease the implementation and use of neural nets. The code base is on github\footnote{\url{https://github.com/tensorflow/tensorflow}}.
In this project Tensorflow is used with Python 2.7 and Jupyter.

\section{Feature Invariance}

\section{Deep Neural Net - RegularNet}
%composed exclusively of regular and strided convolutional layers. While this architecture works well for relatively shallow networks, it becomes increasingly more difficult to train as the network depth increases.
\section{Deep Residual Networks - ResNet}
\section{Densely Connected Convolutional Networks - DenseNet}