\chapter{Implementation}
\label{chp:impl}

This chapter describes some of the implementation details of the project. The idea of the project is to see the classification capabilities of RegularNet, ResNet and DenseNet when the original images has been rotated from 0 to 360 degrees. The implementation of the different nets are implemented in a Jupyter Notebook.

\section{Tensorflow}
TensorFlow is used in this project to create RegularNet, ResNet and DenseNet. 
TensorFlow is an open-source software library created to ease the implementation and use of neural nets. The code base is on github\footnote{\url{https://github.com/tensorflow/tensorflow}}.
%https://www.draw.io/?lightbox=1&highlight=0000ff&edit=_blank&layers=1&nav=1#G0Bw37nIXex8aXYmE1Mll1U1oxUmc
\myFigure{tensorflow.PNG}{Illustration of how TensorFlow is used in this project}{fig:tf}{0.8}

In figure \ref{fig:tf} it is seen how TensorFlow is used in this project. The first part of using TensorFlow is to create the neural net that is needed - how this is done is elaborated in the following chapters. When the neural net is created in TensorFlow the training is ready to commence. 

The Training then yields a model which can be used for classification of images. The model is tested with Cifar-10 test batch. The Output contains probability of the classifications.
\section{Feature Invariance}

Rotation is added to the testing images to see how tolerant the different models are. This section will describe the implementation of the rotation method used to test feature invariance. The implementation of the rotation can be seen in listing \ref{lst:rotation}.

\begin{lstlisting}[language=Python, label=lst:rotation, caption=Rotate the images and calculate the accuracy]
def RotateAndCalc(sess, images, labels):
    #List with all accuracies
    accList = []
    prcList = np.zeros(len(labels))
    
    rotAngle = 10
    curAngle = 10

    #First run for original image
    fAcc,fPrc = CalcTestAccuracy(sess, ConvertImages(images),
	 	labels, False)
    accList.append(fAcc)
    prcList = [x+y for x, y in zip(prcList, fPrc)]

    #Rotate 10 degree
    while curAngle < 360:
        print "Rotating degree " + str(curAngle)
        rotImages = tf.contrib.image.rotate(ConvertImages(images), 
			np.radians(curAngle))

        #Calc accuracy and failures for the rotated images
        sAcc, sPrc = CalcTestAccuracy(sess, rotImages, labels, True)
        prcList = [x+y for x, y in zip(prcList, sPrc)]
        accList.append(sAcc)

        #Increment our curAngle with +rotAngle
        curAngle += rotAngle
       
    #Get failures in percentage
    prcSum = sum(prcList)
    nPrcList = []
    for numb in prcList:
        nPrcList.append((numb / prcSum) * 100)
    
    return accList, nPrcList
\end{lstlisting}

First the accuracy for the original images is calculated, this is the same as having a rotation of 0. Thereafter the images is rotated by 10 degree and the accuracy is calculated for the rotated images. The images will get rotated until they reach a rotation of 350 degree, and therefore a total of 35 accuracies is calculated (from 0 to 350 degree). 360 degree is not included, as this will be the same as a 0 degree rotation. The accuracies is added to a list, which will be used to plot the results. For each of the 35 runs the CalcTestAccuracy function returns a list with the amounts of failure for each label. This could for example be that label 0 (Airplane) failed to classify a specific amount of times. the prcList is after each run updated with these amounts. At last the the amounts of failure for each label is calculated in percentage. This gives a mean for each label for the rotations from 0 to 350 degree. The mean value for each label can be used to see which label is most rotational invariant.
\newline

\mySubFigure{org_image}{rotate_40}{Comparison of the original image and a image which is rotated 40 degree}{Original image}{Image rotated 40 degree}{fig:rot}{fig:orgimg}{fig:rotimg}

Figure \ref{fig:orgimg} shows the frist original image extracted from the CIFAR10 test set. By rotating the original image with 40 degree the image shown in figure \ref{fig:rotimg} is given. The images shown in figure \ref{fig:rot} is blurry, this is because the images in the CIFAR10 set is small, and when plotting them they get stretched. The printed images have been reshaped to a 32x32x3 image to be able to plot them. Three is the depth of the colorspace for the image, which in this case is the RGB colorspace. The images on figure \ref{fig:rot} clearly shows an airplane. 



\section{Deep Neural Net - RegularNet}
%composed exclusively of regular and strided convolutional layers. While this architecture works well for relatively shallow networks, it becomes increasingly more difficult to train as the network depth increases.
\section{Deep Residual Networks - ResNet}
\section{Densely Connected Convolutional Networks - DenseNet}